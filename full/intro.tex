
\chapter{Общая характеристика диссертации}

\section{Актуальность темы исследования}

Беспилотные летательные аппараты (БЛА) мультироторного типа находят все более широкое применение в различных областях деятельности, а миниатюризация и доступность электронных компонентов их бортового оборудования приводит к расширению спектра задач, в решении которых используются такие аппараты.
Интерес исследователей к БЛА обусловлен также и тем, что они являются доступным средством для отработки новых технологий в аэрокосмической отрасли \cite{Otero01}.

С момента создания первых квадрокоптеров интенсивно ведутся исследования в области их динамики и управления, причем количество работ растет с каждым годом, что можно наблюдать по динамике количества публикаций (рис. \ref{pic:res_amount}), посвященных квадрокоптерам, доступных на крупнейшей онлайн платформе научных публикаций «Google Академия».
Однако, несмотря на обилие публикаций в области управляемой динамики квадрокоптеров, остаются перспективные направления, среди которых усовершенствование конструкции БЛА, связанное с увеличением размерности вектора управляющих воздействий.
Конструктивно это достигается, например, с помощью сервоприводов, способных поворачивать роторы с пропеллерами относительно корпуса.
\begin{figure}[h!]
	\centering
	\includegraphics[width=0.95\columnwidth]{res_amount}
	\caption{ -- Динамика количества публикаций, посвященных квадрокоптерам}
	\label{pic:res_amount}
\end{figure}
{
	\vskip 5mm
}

Стандартный квадрокоптер с четырехмерным вектором управляющих воздействий и шестью степенями свободы корпуса аппарата не способен, например, независимо управлять положением и ориентацией.
Это приводит к необходимости дополнительных устройств для наведения камер или лазерных дальномеров, используемых при выполнении ряда стандартных для БЛА задач.
Возможность независимо управлять положением и ориентацией, приобретаемая за счет использования поворотных роторов, влияет не только на работу полезной нагрузки и датчиков, но и на функциональные возможности всей системы в целом.
Согласно работе \cite{Stolc01}, усовершенствованные таким образом квадрокоптеры более устойчивы к возмущениям внешней среды, а также лучше стандартных квадрокоптеров пригодны для вертикального взлета и посадки на неровные поверхности.

Достоинства БЛА с поворотными роторами отмечают и исследователи, работающие над управлением БЛА в экстренных ситуациях (при отказе части двигателей) \cite{Morozov01, Shidar00}.
В работе \cite{Shidar00} обосновывается достижение более высокой скорости за счет выбора оптимальной по отношению к набегающему потоку ориентации, а также более рациональное, по сравнению со стандартными аппаратами, энергопотребление.
В работе \cite{Morozov01} отмечена перспективность конструкции с поворотными роторами, однако ее применение не рассматривается из-за сложности реализации.
Использование поворотных роторов действительно усложняет реализацию контура управления \cite{Ryll01, Falconi01, Segui01, Oosedo01} и не позволяет применять ставшую классической, изящную схему управления \cite{Mellinger01}, однако, есть основания полагать, что предлагаемое в работе аналитическое обращение динамики системы позволит преодолеть некоторую часть возникающих трудностей.

\section{Цели и задачи исследования}

Объектом исследования является система управления движением квадрокоптера с поворотными роторами, предметом исследования -- динамика и методы управления движением квадрокоптера с поворотными роторами.

Цель работы -- выполнить анализ динамики квадрокоптера с поворотными роторами;
разработать алгоритмы управления движением квадрокоптера с поворотными роторами для выполнения различных маневров; разработать алгоритмы управления движением квадрокоптера с поворотными роторами для экстренного сценария -- потере двух смежных двигателей. Для достижения цели работы необходимо решить следующие задачи:
\begin{enumerate}
	\item Разработать математическую модель движения квадрокоптера с поворотными роторами с учетом сил и моментов, действующих на все составные части системы.
	\item Решить задачу обратной динамики и синтезировать контур управления квадрокоптером с поворотными роторами для независимого управления положением и ориентацией аппарата с учетом физических ограничений, накладываемых на исполнительные органы системы управления.
	\item Разработать алгоритмы для идентификации аэродинамических параметров пропеллеров.
	\item Обеспечить обратную связь в контуре управления, реализовать алгоритмы оценки состояния на основе фильтра Калмана.
	\item Реализовать алгоритмы управления квадрокоптером с поворотными роторами в случае аварийного отказа  двигателей, позволяющие аппарату выполнение номинальных задач.
\end{enumerate}
Для решения сформулированных задач используются классические методы механики, управления, вычислительной и высшей математики.

Область исследования соответствует пунктам «Баллистическое проектирование летательных аппаратов различного назначения» и «Динамическое проектирование управляемых летательных аппаратов и исследование динамики их движения» паспорта специальности «05.07.09 – Динамика, баллистика, управление движением летательных аппаратов».

\section{Положения, выносимые на защиту}
Положения, выносимые на защиту:
\begin{enumerate}
	\item Математическая модель управляемой динамики квадрокоптера с поворотными роторами с учетом сил и моментов, действующих на все составные части системы.
	\item Синтез контура управления квадрокоптером с поворотными роторами на основе решения обратной задачи динамики для независимого управления положением и ориентацией аппарата.
	\item Алгоритм для учета физических ограничений, накладываемых на исполнительные органы системы управления.
	\item Алгоритм идентификации аэродинамических параметров пропеллеров.
	\item Алгоритмы оценки состояния квадрокоптера с поворотными роторами.
	\item Алгоритм экстренного управления квадрокоптером с поворотными роторами в случае отказа смежных двигателей.
	\item Выводы и рекомендации, сформулированные в работе.
\end{enumerate}

\section{Степень достоверности и апробация результатов}
Достоверность полученных научных положений, результатов и выводов обеспечивается
соответствием выбранных моделей движения общепринятым стандартам,
адекватностью выбранных методов исследования движения,
проведением численного моделирования полученных аналитических результатов,
а также сопоставлением с результатами,
полученными другими авторами для частных случаев рассматриваемых задач.

Основные научные положения и результаты работы докладывались и обсуждались на 
\begin{itemize}
	\item Всероссийской конференции молодых ученых-механиков, 5 - 15 сентября 2017 г., г. Сочи, «Буревестник» МГУ;
	\item Международной научной конференции «Фундаментальные и прикладные задачи механики», 24 - 27 октября 2017 г., г. Москва;
	\item 60-й Всероссийской научной конференции МФТИ, секция теоретической механики, 20–26 ноября 2017 г., г. Долгопрудный;
	\item 60-й Всероссийской научной конференции МФТИ, секция управления динамическими системами, 20–26 ноября 2017 г., г. Москва;
	\item седьмой международной конференции «Geometry, Dynamics, Integrable Systems», 5-9 июня 2018 г., г. Москва;
	\item XIV Международной конференции «Устойчивость и колебания нелинейных систем управления» (конференция Пятницкого) Россия, Москва, ИПУ РАН, 30 мая -- 1 июня 2018 г.
	\item 14-ой международной конференции «Vibration engineering and technology of machinery», 10-13 сентября 2018 г., г. Лиссабон, Португалия;
	\item Международной конференции «Проблемы механики и управления», 16-22 сентября 2018 г., г. Махачкала;
	\item XXI конференции молодых ученых «Навигация и управление движением», 19–22 марта 2019 г., г. Санкт-Петербург.
	\item XII Всероссийский съезд по фундаментальным проблемам теоретической и прикладной механики, 9 -- 24 августа 2019 г., г. Уфа.
\end{itemize}
По теме исследования опубликованы 14 работ, в том числе 2 представлены в журналах, входящих в базу данных SCOPUS, 1 -- в журнале, входящем в базу данных RSCI, 1 полезная модель к патенту и 3 патентных свидетельства. Cписок можно найти на странице \pageref{list_chapter}.

\section{Научная новизна и практическая значимость работы}
Научная новизна представленных в диссертации результатов заключается в следующем:
\begin{enumerate}
	\item  Разработана математическая модель движения квадрокоптера с поворотными роторами, получено аналитическое решение задачи обратной динамики БЛА.
	\item Синтезирован контур управления квадрокоптером с поворотными роторами на основе решения обратной задачи динамики для независимого управления положением и ориентацией аппарата.
	\item  Проведен анализ аналитического решения задачи обратной динамики  БЛА с поворотными роторами; разработан алгоритм реализации ограничений на компоненты вектора управляющих воздействий, позволяющий учесть физические ограничения исполнительных органов системы управления.
	\item Разработан алгоритм идентификации аэродинамических параметров пропеллеров на основе расширенного фильтра Калмана.
	\item Исследована производительность различных алгоритмов нелинейной фильтрации для оценки состояния БЛА с поворотными роторами.
	\item Разработан алгоритм экстренного управления квадрокоптером с поворотными роторами в случае отказа  смежных двигателей.
\end{enumerate}

Практическая значимость работы состоит в том, что
реализация разработанной в исследовании системы управления позволяет проектировать БЛА с улучшенными относительно стандартных квадрокоптеров летными характеристиками, в том числе обладающих способностью 
выполнять сложные маневры, недоступные стандартным квадрокоптерам, такие, как маневры с требованием независимого управления положением и ориентацией.
Таким образом, расширяются возможности беспилотных летательных аппаратов.
Кроме этого, реализованная в программных алгоритмах динамическая модель и система управления позволяет на предварительном этапе проектирования БЛА определить параметры регулятора и динамики мультироторного робота в зависимости от выбранных комплектующих и других факторов.

\section{Личный вклад автора}
Все результаты, вынесенные на защиту, получены автором самостоятельно.
Также автором самостоятельно проведены численные эксперименты,
подтверждающие основные положения и выводы работы.