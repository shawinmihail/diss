\chapter{Заключение}
Основные результаты диссертации:
\begin{enumerate}
	\item Разработана математическая модель управляемой динамики квадрокоптера с поворотными роторами с учетом сил и моментов, действующих на все составные части системы;
	\item  Получено аналитическое решение задачи обратной динамики БЛА с поворотными роторами;
	\item Синтезирован контур управления квадрокоптером с поворотными роторами для независимого управления положением и ориентацией; 
	\item Разработан алгоритм для учета физических ограничений, накладываемых на исполнительные органы системы управления;
	\item Разработан алгоритм для идентификации основных параметров модели;
	\item Разработаны алгоритмы оценки состояния БЛА с поворотными роторами и проведен их сравнительный анализ.
	\item Разработаны алгоритмы экстренного управления квадрокоптером с поворотными роторами в случае отказа двух смежных двигателей.
\end{enumerate}
По результатам проведенного исследования можно сделать следующие
выводы:
\begin{enumerate}
	\item Расширение размерности вектора управляющих воздействий за счет применения сервоприводов, с помощью которых можно изменять направления тяги каждого из двигателей квадрокоптера, позволяет добиться независимого управления положением и ориентацией БЛА;
	\item Наличие аналитического решения задачи обратной динамики БЛА с поворотными роторами  позволяет ограничить каждую из компонент вектора управляющих воздействий с учетом технических возможностей исполнительных механизмов системы управления, что отражается на маневренных качествах БЛА;
	\item Конструкция квадрокоптера с поворотными роторами позволяет продолжить полет или осуществить безопасную посадку в случае выхода из строя одного или любой пары двигателей;
	\item Квадрокоптер с поворотными роторами способен выполнять задачу преследования подвижного объекта с его одновременным наблюдением с помощью жестко закрепленной на борту камеры.
\end{enumerate}
Таким образом, цели исследования, поставленные в диссертационной работе, достигнуты, и все
поставленные задачи – решены.