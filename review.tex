\documentclass[a4paper,14pt,oneside,openany]{memoir}
\usepackage{amsmath}
\usepackage{ucs} 
\usepackage[utf8]{inputenc} % Включаем поддержку UTF8  
\usepackage[russian]{babel} % Включаем пакет для поддержки русского языка

\date{}  
\author{}
	

\begin{document}
	
	\chapter{Обзор литературы} \label{review}
	
	\section{Летательные аппараты мультироторного типа} \label{review_s1}
	
	История развития летательных аппаратов мультироторного типа началась более ста лет назад, когда братья Бриджет и профессор Ричет создали конструкцию, которую назвали Гироплан №1 ~\cite{Leishman02, Leishman01}. Осенью 1907 года их гироплан вместе с пилотом на борту оторвался от земли менее чем на метр, продемонстрировав практическую возможность использования конструкции такого типа. С тех пор интерес исследователей к мультикоптерам постепенно возрастал и в настоящее время находится на очень высоком уровне во многом благодаря развитию технологий производства комплектующих для аппаратов такого типа и значительному уменьшению габаритов и массы бортовой вычислительной техники и датчиков, необходимых для построения автономной системы управления. Из-за простоты конструкции, маневренности, относительной доступности и дешевизны особенной популярностью сейчас пользуются квадрокоптеры -- беспилотные летательные аппараты мультироторнного типа с четырьмя двигателями. Данные преимущества позволяют использовать такие аппараты как инструмент для отработки новых подходов к построению систем управления и их тестированию, не ограничиваясь только лишь численными методами. Этот факт закономерно привел к тому, что на данный момент существует достаточно большое количество работ, посвященных управляемой динамике беспилотных летательных аппаратов такого типа, значительно отличающихся по поставленным в иследованиях целям, применяемым методам и полученным результатам. В ряде публикаций предлагаются некоторые модификации конструкции стандартного квадрокоптера, которые позволяют повысить лётные характеристики беспилотного летательного аппарата (БЛА) и гарантируют некоторые другие преимущества перед квадрокоптерами стандартной конструкции. Ниже нами рассмотрены встречающиеся в современных публикациях основные подходы, посвященные проектированию систем управления как стандартных так и модифицированных конструкций БЛА, в основе которых лежат ПИД-регуляторы, линейно-квадратичные регуляторы, скользящие режимы управления, адаптивное управление, робастное управление и другие методы современной теории управления.

	\section{Конструкция и основные принципы движения квадрокоптера}
	
	Основными элементами конструкции стандартного квадрокоптера являются корпус и четыре двигателя с прикрепленными к ним пропеллерами. В зависимости от взаимного расположения двигателей и собственной оси продольного движения выделяют две стандартных схемы: \textit{cross}-схему и \textit{plus}-схему~\cite{Bashi01}.
	\\
	
	КАРТИНКА СХЕМЫ И ВРАЩЕНИЯ РОТОРОВ
	\\
	
	Пропеллеры, расположенные на смежных лучах вращаются в разные стороны, благодаря чему возможна стабилизация по углу рысканья, при этом тяга каждого двигателя направленна одинаково. Вертикальное движение аппарата обусловлено изменением общей тяги всех двигателей. Горизонтальное движение совершается за счет изменения направления вектора тяги вследствие наклона корпуса БЛА. Таким образом, независимыми параметрами движения являются положение и угол рысканья квадрокоптера. 
	
	\section{Математическая модель}
		
	Обычно, поступательное движение малых беспилотных летательных аппаратов рассматривают в инерциальной системе координат (назовем её {$I$}), связанной с поверхностью Земли. С учетом относительно небольших характерных расстояний и времен полета БЛА движением и кривизной поверхности Земли принебрегают.
	
	Существуют два наиболее распространенных способа связать координатные оси с Землей: {$NED$} --  ось \textbf{$X_I$} направлена на север, ось \textbf{$Y_I$} -- на восток, а ось \textbf{$Z_I$} -- к центру Земли; {$ENU$} -- ось \textbf{$X_I$} направлена на восток, ось \textbf{$Y_I$} -- на север, а ось \textbf{$Z_I$} -- от центра Земли. Положение БЛА описывается радиус-вектором центра масс аппарата \boldmath{$r_I$},записанном в выбранной инерциальной системе координат.

	Корпус квадрокоптера в рамках задачи управления движением обычно считают твердым телом. Для описания ориентации объекта в пространстве используют собственную систему координат, начало которой связывают с центром масс БЛА, а оси направляют по осям инерциальной системы.
	Ориентация объекта в пространстве довольно часто описывается с помощью углов Эйлера {($\phi$ $\theta$ $\psi$)} или кватерниона ориентации {$q_{BI}$}, который задает положение собственного базиса {$B$} относительно базиса {$I$}. Иногда встречаются и другие подходы для описания ориентации...
	\\
		
	КАРТИНКА КВАДРОКОПТЕР И СИСТЕМЫ КООРДИНАТ
	\\
	
	Поступательное движение БЛА определяется силами гравитации, аэродинамического сопротивления и тягой пропеллеров. Иногда в моделях могут присутствовать и другие возмущения. В их отсутствии уравнения поступательного движения имеют вид
	
	\begin{equation} \label{eq:common_traslational_motion}
	\ddot{r} = \frac{1}{m}(F_g + F_{aero} + F_{thr}).
	\end{equation}
	
	Основными параметрами поступательной динамики БЛА являются общая масса {$m$} конструкции, ускорение свободного падения \textbf{{$g$}}, плотность среды {$\rho_{air}$}, аэродинамические свойства корпуса аппарата и пропеллеров. Сила тяжести определяется выражением
	
	\begin{equation} \label{eq:gravity_force}
	F_g = mg.
	\end{equation}
	
	Аэродинамические свойства корпуса БЛА в простых моделях определются площадью лобового сечения корпуса аппарата {$S_{\perp}$} и аэродинамическими константами. Выражение для аэродинамической силы можно записать, как
	 
	\begin{equation} \label{eq:aerodynamic_force}
	F_{aero} = - \frac{1}{2} \rho_{air} C S_{\perp} |\dot{r}| \dot{r}.
	\end{equation}
	
	Силу тяги пропеллеров во многих работах определяют через квадрат их угловой скорости $\omega$ и аэродинамический коэффициент $k$.
	
	\begin{equation} \label{eq:thrust_force}
	F_{thr} = \sum_{i=1}^{4}{ {F_{i}^{ext} }}
	\end{equation}
	
	\begin{equation} \label{eq:rotor_ext_force}
	F_{i}^{ext} = k \omega^2_i z_i.
	\end{equation}
	
	Здесь $z_i$ -- ось вращения $i$-того пропеллера.
	
    Кроме этих сил, в некоторых работах также рассматриваются ...
    
    Вращательное движение корпуса БЛА определяется моментами сил, которые создают двигатели с пропеллерами и гироскопическим моментом самого корпуса. Одним из основных параметров вращательной динамики является тензор инерции $J$ корпуса квадрокоптера, который записывают в главных осях. Часто БЛА проектируют таким образом, что главные оси инерции корпуса совпадают с осями собственной системы координат.
    
   	\begin{equation} \label{eq:common_rotational_motion}
   	\sum_{i=1}^{4}{\tau_i} = J\dot{\Omega} + J{\Omega} \times \Omega.
   	\end{equation}
    
     При рассмотрении моментов сил, действующих со стороны двигателей, важным параметром является тензор инерции каждого ротора с пропеллером $J_i$ и аэродинамические свойства пропеллеров, отвечающие за связь скорости вращения пропеллера и действующего на него внешнего момента. Этот момент связывают с квадратом оборотов пропеллера с помощью аэродинамического коэффициента, который мы обозначим $b$. Момент сил, действующий на корпус аппарата со стороны одного ротора с пропеллером может быть записан, как
     
     
   	\begin{equation} \label{eq:rotor_torque}
   	\tau_i = \tau_i^{ext} - J_i \dot{\omega_i} - J_i{\omega_i} \times \omega_i
   	\end{equation}
   	
 	\begin{equation} \label{eq:rotor_ext_torque}
	\tau_i^{ext} = -b \omega^2_i z_i.
 	\end{equation}
   	
     Иногда модель дополняется другими возмущающими моментами...
     
     При синтезе контура управления квадрокоптера стандартной конструкции в качестве компонентов вектора управления выбирают скорости вращения его двигателей или некоторые их функции. С учетом выражений (\ref{eq:rotor_ext_force}), (\ref{eq:rotor_ext_torque}) удобно сформировать управляющий вектор из квадратов скоростей вращения пропеллеров
 
   	\begin{equation} \label{eq:common_control_vector}
	 u_i = w_i^2.
   	\end{equation}
     Также, нередко, компоненты вектора управляющего воздействия формируют таким образом, чтобы три из них соответствовали моментам сил или угловым ускорениям, направленным вдоль собственных осей корпуса БЛА, а оставшаяся компонента -- общей тяге всех роторов или ускорению вдоль собственной оси {Z}. Конкретные выражения для компонент будут зависить от рассматривоей схемы квадрокоптера и способа нумерации его двигателей.
     
     Цели управления БЛА могут быть сформулированы по-разному, начиная от приведения его центра масс в некоторое наперед заданное статичное положение, заканчивая преследованием движущегося объекта. Однако, в связи с невозможностью независимого управляения положением и ориентацией квадрокоптера стандартной конструкции, задачи управления сводятся, в основном, к управлению его положением.
     
     Увелечение размерности вектора управляющего воздействия, например, за счет изменения угла атаки лопастей пропеллера или изменения ориентации двигателей относительно корпуса, может разнообразить множество возможных постановок задач управления. Среди них ...
     
	Далее мы детально рассмотрим те техники и подходы, которые применяются для решения поставленных задач. 
	
	\section{Управление с использованием ПИД-регуляторов}
	
	 ПИД-регуляторы применяются в более чем 90\% промышленных систем управления ~\cite{Astrom01}, в том числе широко используются в системах управления БЛА.  Такая популярность связана с их простой и понятной реализацией, а также рядом важных особенностей, таких, как способность устранять устойчивые ошибки благодаря интегральной составляющей и прогнозировать состояние управляемой системы в будущем с помощью дифференциальной составляющей. Помимо этого, алгоритмы управления, использующие ПИД-регуляторы обычно не требуют больших вычислительных ресурсов. Однако, существует ряд проблем при применении данной техники к построению систем управления квадрокоптероми, включающие нелинейности, связанные с математической моделью и неточностями моделирования динамики. Поэтому применение ПИД-регуляторов для квадрокоптеров может ограничить их производительность ~\cite{Zulu01}.
	
	Работа ПИД-регулятора основана на вычислении текущей ошибки положения в координатном пространстве $e$, которая является разностью целевого состояния и оценки текущего состояния. Затем, управляющее воздействие вычисляется, как
	\begin{equation} \label{eq:pid_common}
	u(t) = K_Pe(t) + K_I\int_0^t e(t) dt + K_D \frac{de(t)}{dt},
	\end{equation}
	где $K_P$, $K_I$ и $K_D$ -- пропорциональный, интегральный и дифференциальный коэффициенты соответственно.
	\\
	
	КАРТИНКА ПИД СХЕМА
	\\
	
	В работе ~\cite{Li01} применена схема с использованием ПИД-регулятора для управления положением и ориентацией БЛА. Рассматриваемый аппарат был спроектирован по \textit{plus}-схеме. Ориентация аппарата описана углами Эйлера. Динамическая модель включает в себя все основные силы и моменты, действующие на аппарат, без каких-либо дополнительных возмущений. Вектор состояния включает в себя скорость центра масс БЛА, углы его ориентации и их производные по времени. Компоненты вектора управления являются функциями от квадратов скоростей вращения пропеллеров; первая компонента отвечает за общую тягу всех двигателей, остальные три -- за моменты сил, действующие вдоль собственных координатных осей. Используя метод малых возмущений, авторы приводят модель к линейной, находят передаточные функции по всем каналам управления, которые затем используют для построения \textit{Simulink}-модели. Параметры модели соответствуют небольшому летательному аппарату с массой чуть больше одного килограмма. В работе подобраны коэффициенты ПИД-регулятора, при которых БЛА удается стабилизировать свою позицию и ориентацию около целевой точки. Далее в работе описан прототип летательного аппарата, реализующий предложенные алгоритмы. Проведены летные испытания, ошибка по углам ориентации не превысила пяти градусов.
	
	Последовательное применение ПД-регуляторов для управления позицией и ориентацией используется у \textit{Mellinger D., Kumar V.} ~\cite{Mellinger01}. Рассматриваемый БЛА также спроектирован по \textit{plus}-схеме. Модель содержит все основные силы и моменты, помимо аэродиномического сопротивления воздуха. Ориентация представлена матрицами поворота. В некоторых выражениях для наглядности используются углы Эйлера. Приведены аргументы в пользу использования простой модели двигателей [\ref{eq:rotor_ext_force}, \ref{eq:rotor_ext_torque}]. Компоненты вектора управляющего воздействия отвечают за тягу и моменты, действующие на корпус БЛА со стороны приводов. Целью управления является отслеживание аппаратом траектории и целевого угла рысканья. Авторы показывают, что модель является дифференциально плоской относительно выхода
	\begin{equation} \label{eq:mellinger_flat_output}
	\sigma = (r,\psi)^T,
	\end{equation}
	то есть вектор состояния и вектор управляющих воздействий может быть записан, как функция выхода  \ref{eq:mellinger_flat_output} и конечного числа его производных.
	Это позволяет в дальнейшем строить оптимальные траектории в пространстве вектора выхода.
	
	Синтезирован двухуровненвый контур управления. На первом уровне вычисляется ошибка позиционирования БЛА и используется ПД-регулятор для определения вектора целевой тяги квадрокоптера. На основе направления целевой тяги и целевого угла рысканья расчитывается целевая ориентация аппарата; для приведения корпуса аппарата в необходимое положение также используется ПД-регулятор.
	
	В заключительной части работы рассматривается вопрос построения траекторий. Авторы строят кусочно-полиминиальную кривую через набор точек маршрута, минимизируя интеграл квадрата нормы второй производной ускорения центра масс аппарата и интеграл квадрата нормы второй производной угла его рысканья. Затем  авторы показывают, как можно локализовать некоторые участки траектории в прямоугольной области, для того, чтобы избежать их пересечений с возможными препятствиями.
	
	Производительность алгоритма демонстрируется в натурном эксперименте. Траекторией БЛА является окружность, скорость аппарата превышает два с половиной метра в секунду, а углы крена тангажа достигают сорока градусов. Ошибка позиционирования не превышает пятнадцати сантиметров, однако ошибка ориентации авторами не приводится.
	
	Важной особенностью работы является учет нелинейной природы динамики квадрокоптера при синтезе контура управления. Это позволяет сохранять устойчивость управления при значительных отклонениях корпуса аппарата по углам тангажа и рысканья.
	
	Исследование относительной эффективности применения ПИД-регулятора для стабилизации ориентации корпуса квадрокоптера проводят \textit{Bouabdallah S., Noth A., Siegwart R.}~\cite{Bouabdallah01}. Кроме стандартных возущений, в модели присутствуют выражения для динамики бесколлекторных моторов. Для синтеза контура управления авторы принебрегают гироскопическими и некоторыми другими незначительными эффектами, упрощают и лианерезуют модель около текущего положения. Ошибка ориентации, равная разнице целевого и текущего положения, записанных в углах Эйлера, подается на вход ПИД-регулятора, затем вычисляется вектор управления. Проведены эксперименты в \textit{Matlab Simulink} и на тестовом стенде. Авторы отмечают хорошую производительность флгоритмов управления, принимают во внимание эффекты, возникающие из-за более подробного моделирования динамики роторов и отмечают необходимость быстрого реагирования приводов робота на управление.
	
	Затем авторы синтезируют второй контур управления, в основе которого лежит линейно-квадратичный регулятор и сравнивают получнные результаты.
		
	
	\section{Управление с использованием ЛКР-регуляторов}
	
	Линейно-квадратичный регулятор -- один из видов оптимальных регуляторов, использующий квадратичный функционал в качестве функции стоимости.
	
	
	\bibliography{bibliography/citations.bib}{}
	\bibliographystyle{plain}
\end{document}