

\chapter{Обзор литературы} \label{review}
\section{Летательные аппараты мультироторного типа} \label{review_s1}

История развития летательных аппаратов мультироторного типа началась более ста лет назад, когда братья Бриджет и профессор Ричет создали конструкцию, которую назвали Гироплан №1 ~\cite{Leishman02, Leishman01}. Осенью 1907 года их гироплан вместе с пилотом на борту оторвался от земли менее чем на метр, продемонстрировав практическую возможность использования конструкции такого типа. С тех пор интерес исследователей к мультикоптерам постепенно возрастал и в настоящее время находится на очень высоком уровне во многом благодаря развитию технологий производства комплектующих для аппаратов такого типа и значительному уменьшению габаритов и массы бортовой вычислительной техники и датчиков, необходимых для построения автономной системы управления. Из-за простоты конструкции, маневренности, относительной доступности и дешевизны особенной популярностью сейчас пользуются квадрокоптеры -- беспилотные летательные аппараты мультироторнного типа с четырьмя двигателями. Данные преимущества позволяют использовать такие аппараты как инструмент для отработки новых подходов к построению систем управления и их тестированию, не ограничиваясь только лишь численными методами. Этот факт закономерно привел к тому, что на данный момент существует достаточно большое количество работ, посвященных управляемой динамике беспилотных летательных аппаратов такого типа, значительно отличающихся по поставленным в иследованиях целям, применяемым методам и полученным результатам. В ряде публикаций предлагаются некоторые модификации конструкции стандартного квадрокоптера, которые позволяют повысить лётные характеристики беспилотного летательного аппарата (БЛА) и гарантируют некоторые другие преимущества перед квадрокоптерами стандартной конструкции. Ниже нами рассмотрены встречающиеся в современных публикациях основные подходы, посвященные проектированию систем управления как стандартных так и модифицированных конструкций БЛА, в основе которых лежат ПИД-регуляторы, линейно-квадратичные регуляторы, скользящие режимы управления, адаптивное управление, робастное управление и другие методы современной теории управления.

\section{Конструкция и основные принципы движения квадрокоптера}

Основными элементами конструкции стандартного квадрокоптера являются корпус и четыре двигателя с прикрепленными к ним пропеллерами. В зависимости от взаимного расположения двигателей и собственной оси продольного движения выделяют две стандартных схемы: \textit{cross}-схему и \textit{plus}-схему~\cite{Bashi01}.
\\

КАРТИНКА СХЕМЫ И ВРАЩЕНИЯ РОТОРОВ
\\

Пропеллеры, расположенные на смежных лучах вращаются в разные стороны, благодаря чему возможна стабилизация по углу рысканья, при этом тяга каждого двигателя направленна одинаково. Вертикальное движение аппарата обусловлено изменением общей тяги всех двигателей. Горизонтальное движение совершается за счет изменения направления вектора тяги вследствие наклона корпуса БЛА. Таким образом, независимыми параметрами движения являются положение и угол рысканья квадрокоптера ~\cite{Salih01}. 

\section{Математическая модель}
	
Обычно, поступательное движение малых беспилотных летательных аппаратов рассматривают в инерциальной системе координат (назовем её {$I$}), связанной с поверхностью Земли. С учетом относительно небольших характерных расстояний и времен полета БЛА движением и кривизной поверхности Земли принебрегают.

Существуют два наиболее распространенных способа связать координатные оси с Землей: {$NED$} --  ось \textbf{$X_I$} направлена на север, ось \textbf{$Y_I$} -- на восток, а ось \textbf{$Z_I$} -- к центру Земли; {$ENU$} -- ось \textbf{$X_I$} направлена на восток, ось \textbf{$Y_I$} -- на север, а ось \textbf{$Z_I$} -- от центра Земли. Положение БЛА описывается радиус-вектором центра масс аппарата \bm{$r_I$},записанном в выбранной инерциальной системе координат. Скорость определяется, как
\begin{equation} \label{eq:velocity}
\bm{v_I} = \dot{\bm{r}}.
\end{equation}

Корпус квадрокоптера в рамках задачи управления движением обычно считают твердым телом. Для описания ориентации объекта в пространстве используют связанную с телом систему координат $B$, начало которой совпадает с центром масс БЛА, а оси направлены по главным центральным осям инерции корпуса. Текущее положение базиса B относительно I может быть описано матрицей поворота $\bm{R}_{IB}$, в том смысле, что разложения произвольного вектора $\bm{x}$, записанного в этих базисах, связаны соотношением
\begin{equation} \label{eq:rotmx}
\bm{x}_I = \bm{R}_{IB}\bm{x}_B.
\end{equation}
Матрица поворота связана с компонентами угловой $\bm{\Omega}_B$ скорости следующим соотношением

\begin{equation} \label{eq:angvel_rotmx}
\dot{\bm{R}}_{IB} = \bm{R}_{IB} [\bm{\Omega}_B]_{\times},
\end{equation}
где
\begin{equation} \label{eq:hat_operator}
[\bm{\Omega}]_{\times} =
\begin{bmatrix}
0            & -\bm{\Omega}^z   & \bm{\Omega}^y \\
\bm{\Omega}^z     & 0           &-\bm{\Omega}^x\\
-\bm{\Omega}^y    & \bm{\Omega}^x    & 0
\end{bmatrix}.
\end{equation}

Иногда для описания ориентации БЛА используют углы конечного поворота (иногда их называют углами Эйлера вне зависимости от последовательности), которые задают положение базиса B относительно базиса I через три  последовательных поворота вокруг осей связанной системы координат. Наиболее часто используются так называемые самолетные углы: последовательные повороты вокруг оси $Z_B$ на угол $\psi$, $Y_B$ на угол $\theta$, $X_B$ на угол $\phi$ будут определять углы рысканья, крена и тангажа. Им соответствует матрица поворота

\small
\begin{equation*} \label{eq:eul_to_rotmx}
\begin{aligned}
&	\bm{R}_{IB} = \\
&\begin{bmatrix}
cos(\psi)cos(\theta) & cos(\psi)sin(\theta)sin(\phi) - sin(\psi)cos(\phi) & cos(\psi)sin(\theta)cos(\phi) + sin(\psi)sin(\phi) \\
sin(\psi)cos(\theta) & sin(\psi)sin(\theta)sin(\phi) - cos(\psi)cos(\phi) & sin(\psi)sin(\theta)cos(\phi) - cos(\psi)sin(\phi) \\
-sin(\theta)         & cos(\psi)sin(\phi)                                 & cos(\psi)cos(\phi)\\
\end{bmatrix}.
\end{aligned}
\end{equation*}
\normalsize


Помимо матриц и углов конечного поворота для описания ориентации часто используются кватернионы ~\cite{Amelkin01} -- четырехмерные гиперкомплексные числа, которые можно представить в виде формальной суммы скалярной и векторной частей
	 \begin{equation} \label{eq:quat_def}
	 Q = q_0 + \bm{q}.
	 \end{equation}
	 Для кватернионов определены операции сопряжения
	 \begin{equation} \label{eq:quat_dual}
	 \tilde{Q} = q_0 - \bm{q}
	 \end{equation}
	 и умножения
\begin{equation} \label{eq:quat_mult}
Q \circ P = q_0 p_0 - (\bm{q} \cdot \bm{p})
+ q_0 \bm{p} + p_0 \bm{q} + \bm{q} \times \bm{p}
\end{equation}
Кватернион ориентации  $q_{IB}$ определяет положение собственного базиса {$B$} относительно базиса {$I$} в том смысле, что разложения произвольного вектора $\bm{x}$, записанного в этих базисах, связаны соотношением
	\begin{equation} \label{eq:quat}
	\bm{x}_I = q_{IB} \circ \bm{x}_B \circ \tilde{q}_{IB}.
	\end{equation}

	Угловая скорость и кватернион ориентации связаны уравнением Пуассона
	\begin{equation} \label{eq:puasson}
	\dot{q}_{IB} = \frac{1}{2} {q}_{IB} \circ \bm{\Omega}_B.
	\end{equation}
	Кватернион ориентации эквивалентен матрице поворота
	\begin{equation} \label{eq:quat_to_rotmx}
	\bm{R} = ({q_0}^2 - \bm{q}^T \bm{q}) E_{3 \times 3} + 2 \bm{q}^T \bm{q} - 2 {q_0} [\bm{q}]_{\times},
	\end{equation}
	где $E_{n \times n}$ -- едичная матрица размерности $n$.
\\
КАРТИНКА КВАДРОКОПТЕР И СИСТЕМЫ КООРДИНАТ
\\
Движение центра масс БЛА определяется силами гравитации, аэродинамического сопротивления и тягой пропеллеров. Иногда в моделях могут присутствовать и другие возмущения (НАПРИМЕР). В их отсутствии уравнения поступательного движения имеют вид

\begin{equation} \label{eq:common_traslational_motion}
m \ddot{\bm{r}} = \bm{F}_g + \bm{F}_{aero} + \bm{F}_{thr}.
\end{equation}

Основными параметрами, определяющими движение центра масс БЛА являются общая масса {$m$} конструкции, ускорение свободного падения \bm{$g$}, плотность среды {$\rho_{air}$}, аэродинамические свойства корпуса аппарата и пропеллеров. Сила тяжести определяется выражением

\begin{equation} \label{eq:gravity_force}
\bm{F}_g = m\bm{g}.
\end{equation}

Аэродинамические свойства корпуса БЛА определются площадью лобового сечения корпуса аппарата {$S_{\perp}$} и аэродинамическими константами. Выражение для аэродинамической силы можно записать, как ~\cite{Biard01}

\begin{equation} \label{eq:aerodynamic_force}
\bm{F}_{aero} = - \frac{1}{2} \rho_{air} C S_{\perp} |\dot{\bm{r}}| \dot{\bm{r}}.
\end{equation}

Силу тяги пропеллеров обычно определяют через квадрат их оборотов $\tilde\omega$ и аэродинамический коэффициент $k$ ~\cite{Falconi01}

\begin{equation} \label{eq:thrust_force}
\bm{F}_{thr} = \sum_{i=1}^{4}{ { k \tilde\omega^2_i \bm{z}_i}.}
\end{equation}

Здесь $\bm{z}_i$ -- ось вращения $i$-того пропеллера.
 
Вращательное движение корпуса БЛА определяется моментами сил, которые создают двигатели с пропеллерами и гироскопическим моментом самого корпуса.

  	\begin{equation} \label{eq:common_rotational_motion}
  	\sum_{i=1}^{4}{\bm{\tau}_{Bi}} = \bm{J}_B\dot{\bm{\Omega}_B} + \bm{J}_B{\bm{\Omega}_B} \times \bm{\Omega}_B,
  	\end{equation}

где $\bm{J}_B$ -- тензор инерции корпуса БЛА, записанный в его главных осях.

 Внешний момент, действующий на пропеллер, связывают с квадратом его оборотов с помощью аэродинамического коэффициента $b$ ~\cite{Ryll01}
	
	\begin{equation} \label{eq:rotor_ext_torque}
\bm{\tau}_{Bi} = -b \tilde{\omega}^2_i \bm{z_i}.
	\end{equation}
  	
 При синтезе контура управления квадрокоптера стандартной конструкции в качестве компонентов вектора управления выбирают обороты его двигателей или некоторые их функции ~\cite{Lee02, Sharifi01, Luukkonen01, Bemporad01}.  С учетом выражений (\ref{eq:thrust_force}), (\ref{eq:rotor_ext_torque}) удобно сформировать вектор управляющих воздействий из квадратов скоростей вращения пропеллеров

  	\begin{equation} \label{eq:common_control_vector}
u_i = \tilde{w}_i^2.
  	\end{equation}
 Конкретные выражения для компонент будут зависеть от рассматриваемой схемы квадрокоптера и способа нумерации его двигателей.
 
 \section{Постановка задачи управления}
 Цели управления БЛА могут быть сформулированы по-разному,
 начиная от приведения его центра масс в некоторое наперед заданное статичное положение
 ~\cite{Huynh01, Yuskin01}
 или обеспечения требуемой ориентации и высоты
 ~\cite{Domingos01, Wang01, Gheorghita01, Lukmana01, Zabko01},
 и заканчивая наблюдением за внешними объектами
 ~\cite{Rodriguez01, Kendall01, Razinkova01}
 и построением пространственных формаций
 ~\cite{Ali01, Zhao01, Preiss01}.
 
 Увелечение размерности вектора управляющего воздействия, например, за счет изменения угла атаки лопастей пропеллера ~\cite{Cutler01, Cutler02}  или изменения ориентации двигателей относительно корпуса ~\cite{Sridhar02, Kumar02} позволяет расширить маневренные возможности БЛА и усложнить поставленные перед ними задачи. Например, в работе ~\cite{Ryll02} в качестве задачи управления выбрано отслеживание аппаратом произвольной траетории в пространстве при независимом управлении ориентацией его корпуса. Подобная задача была решена в исследовании ~\cite{Kaufman01} для БЛА с шестью двигателями с использованием пропеллеров с переменной геометрией. Применение таких пропеллеров для экстремальных маневров, включая фигуры высшего пилотажа, показали в работе ~\cite{Cutler02}.
 
 Далее мы детально рассмотрим те техники и подходы, которые применяются для решения поставленных задач управления. 

\section{Управление с использованием ПИД-регуляторов}

ПИД-регуляторы широко используются в системах управления БЛА.
Такая популярность связана с их простой и понятной реализацией, а также рядом важных особенностей, таких, как способность устранять статические ошибки благодаря интегральной составляющей и прогнозировать состояние управляемой системы с помощью дифференциальной составляющей  ~\cite{Astrom01}.
Помимо этого, алгоритмы управления, использующие ПИД-регуляторы обычно не требуют больших вычислительных ресурсов.
Однако, существует ряд проблем при применении данной техники к построению систем управления квадрокоптероми, включающие нелинейности, связанные с математической моделью и неточностями моделирования динамики.
Поэтому применение ПИД-регуляторов для квадрокоптеров может ограничить их производительность ~\cite{Zulu01}.

Работа ПИД-регулятора основана на вычислении текущей ошибки положения в координатном пространстве $\bm{e}$, которая является разностью целевого состояния и оценки текущего состояния. Затем, управляющее воздействие вычисляется, как
\begin{equation} \label{eq:pid_common}
\bm{u}(t) = K_P \bm{e}(t) + K_I\int_0^t \bm{e}(t) dt + K_D \frac{d\bm{e}(t)}{dt},
\end{equation}
где $K_P$, $K_I$ и $K_D$ -- пропорциональный, интегральный и дифференциальный коэффициенты соответственно.

В работе ~\cite{Li01} применена схема с использованием ПИД-регулятора для управления положением и ориентацией БЛА.
Рассматриваемый аппарат спроектирован по \textit{plus}-схеме.
Ориентация аппарата описана углами Эйлера.
Динамическая модель включает в себя все основные силы и моменты (\ref{eq:common_traslational_motion}) - (\ref{eq:rotor_ext_torque}), действующие на аппарат, без каких-либо дополнительных возмущений.
Вектор состояния включает в себя скорость центра масс БЛА, углы его ориентации и их производные по времени.
Компоненты вектора управления являются функциями от квадратов скоростей вращения пропеллеров; первая компонента отвечает за общую тягу всех двигателей, остальные три -- за моменты сил, действующие вдоль собственных координатных осей.
Используя метод малых возмущений, авторы линеаризуют модель в окрестности текущего положения, находят передаточные функции по всем каналам управления, которые затем используют для построения \textit{Simulink}-модели.
Параметры модели соответствуют небольшому летательному аппарату с массой чуть больше одного килограмма.
В работе подобраны коэффициенты ПИД-регулятора, при которых БЛА удается стабилизировать свою позицию и ориентацию около целевой точки.
Далее в работе описан прототип летательного аппарата, реализующий предложенные алгоритмы.
Проведены летные испытания, ошибка по углам ориентации не превысила пяти градусов.

Последовательное применение ПД-регуляторов для управления позицией и ориентацией используется  у ~\cite{Mellinger01}. Рассматриваемый апарат также спроектирован по \textit{plus}-схеме. Модель содержит все основные силы и моменты
(\ref{eq:common_traslational_motion}) - (\ref{eq:rotor_ext_torque}),
помимо аэродиномического сопротивления воздуха.
Ориентация описывается углами Эйлера и с помощью матриц поворота.
Приведены некоторые аргументы в пользу использования простой модели двигателей
(\ref{eq:thrust_force}, \ref{eq:rotor_ext_torque}).
Компоненты вектора управляющего воздействия отвечают за тягу и моменты, действующие на корпус БЛА со стороны приводов:
\begin{equation} \label{eq:mellinger_control_vector}
	\begin{aligned}
	\bm{u} =
	\begin{bmatrix}
	k & k & k & k\\
	0 & bL & 0 & -bL\\
	-bL & 0 & bL & 0\\
	b & -b & b & -b
	\end{bmatrix}
	\begin{bmatrix}
	\omega^{2}_{1}\\
	\omega^{2}_{2}\\
	\omega^{2}_{3}\\
	\omega^{2}_{4}
	\end{bmatrix},
	\end{aligned}
\end{equation}
где $L$ -- расстояние от центра масс БЛА до осей вращения роторов.
Вектор состояния включает в себя позицию, ориентацию, а также линейную и угловую скорости аппарата
\begin{equation} \label{eq:mellinger_state}
\bm{x} = [\bm{r}_I, \phi, \theta, \psi, \bm{v}_I, \bm{\Omega}_B].
\end{equation}

Целью управления является отслеживание аппаратом траектории в пространстве и заданного угла рысканья. Авторы показывают, что модель является дифференциально плоской ~\cite{Belinskaya01, Nieuwstadt01} относительно выхода
\begin{equation} \label{eq:mellinger_flat_output}
\bm{\sigma} = (\bm{r},\psi)^T,
\end{equation}
то есть вектор состояния (\ref{eq:mellinger_state}) и вектор управляющих воздействий (\ref{eq:mellinger_control_vector}) может быть записан, как функция выхода  (\ref{eq:mellinger_flat_output}) и конечного числа его производных. Данный факт позволяет в дальнейшем строить оптимальные траектории в пространстве вектора выхода.

Синтезирован двухуровненвый контур управления. На первом уровне вычисляется ошибка позиционирования БЛА и используется ПД-регулятор для определения вектора целевой тяги квадрокоптера:
\begin{equation} \label{eq:mellinger_pos_err}
\bm{e}_r = \bm{r}^{0} - \bm{r},
\end{equation}
\begin{equation} \label{eq:mellinger_vel_err}
\bm{e}_v = \bm{v}^{0} - \bm{v},
\end{equation}
\begin{equation} \label{eq:mellinger_pos_reg}
\bm{F}^0 = K_r \bm{e}_r + K_v \bm{e}_v + m \bm{g} + m \ddot{\bm{r}}^0,
\end{equation}
где $\bm{r}^{0}$, $\bm{v}^{0}$, $\ddot{\bm{r}}^{0}$, $\bm{F}^{0}$ -- целевые координата, скорость, ускорение и общая сила тяги аппарата, а $K_r$ и $K_v$ -- некоторые положительно определенные матрицы.
Затем, положив $ || \bm{F}^{0} || > 0$
(что равносильно запрету на свободное падение БЛА),
авторы вычисляют первую компоненту вектора управляющих воздействий (\ref{eq:mellinger_control_vector}),
как
\begin{equation} \label{eq:mellinger_u1}
u_1 = \bm{F}^{0} \cdot \bm{z}_B,
\end{equation}
и определяют целевую ориентацию при известном угле рысканья из условия
\begin{equation} \label{eq:mellinger_Rdes}
\bm{R}_{IB}^0 [0,0,1]^T = \frac{\bm{F}^{0}}{||\bm{F}^{0}||}.
\end{equation}
На основе ошибки ориентации
\begin{equation} \label{eq:mellinger_eR}
\bm{e}_R = \frac{1}{2}
\Big[
(\bm{R}_{IB}^0)^T 	\bm{R}_{IB} -
(\bm{R}_{IB})^T \bm{R}_{IB}^0
\Big]_\vee,
\end{equation}
где оператор $[...]_\vee$ является обратным преобразованием к (\ref{eq:hat_operator}), и ошибки по угловой скорости вычисляются оставшиеся компоненты вектора управляющих воздействий
\begin{equation} \label{eq:mellinger_att_reg}
[u_2, u_3, u_4]^T = K_R \bm{e}_R + K_{\Omega} \bm{e}_{\Omega}.
\end{equation}
Затем, авторы вычисляют значения оборотов каждого из двигателей на основе выражения (\ref{eq:mellinger_control_vector}).
	
В заключительной части работы рассматривается вопрос построения траекторий. Авторы строят кусочно-полиминиальную кривую через набор точек маршрута, минимизируя интеграл квадрата нормы второй производной ускорения центра масс аппарата и интеграл квадрата нормы второй производной угла его рысканья. Затем  авторы показывают, как можно локализовать некоторые участки траектории в прямоугольной области, для того, чтобы избежать их пересечений с возможными препятствиями.

Производительность алгоритмов демонстрируется в натурном эксперименте. Траекторией БЛА является окружность, скорость аппарата превышает два с половиной метра в секунду, а углы крена тангажа достигают сорока градусов. Ошибка позиционирования не превышает пятнадцати сантиметров, однако ошибка ориентации авторами не приводится. Важной особенностью работы является учет нелинейной природы динамики квадрокоптера при синтезе контура управления. Это позволяет сохранять устойчивость управления при значительных отклонениях корпуса аппарата по углам тангажа и рысканья.

Исследование относительной эффективности применения ПИД регулятора для стабилизации ориентации корпуса квадрокоптера проводят авторы работы ~\cite{Bouabdallah01}. Кроме стандартных возущений, в модели присутствуют выражения для динамики бесколлекторных двигателей. Для синтеза контура управления авторы принебрегают гироскопическими и некоторыми другими эффектами, упрощают и линеаризуют модель в окрестности точки соотвествующей неподвижному зависанию БЛА в воздухе. Ошибка ориентации, равная разнице целевого и текущего положения, записанных в углах Эйлера, подается на вход ПИД-регулятора, затем вычисляется вектор управления. Проведены вычислительные эксперименты и эксперименты на тестовом стенде. Авторы отмечают хорошую производительность алгоритмов управления, принимают во внимание эффекты, возникающие из-за более подробного моделирования динамики роторов и отмечают необходимость быстрого реагирования приводов робота на управление.

Затем происходит синтез второго контура управления, в основе которого лежит линейно-квадратичный регулятор (ЛКР-регулятор) и сравнение полученных результатов.
	

\section{Управление с использованием ЛКР-регуляторов}

Линейно-квадратичный регулятор -- один из видов оптимальных регуляторов, который применяется в линейных системах вида
\begin{equation} \label{eq:linear_dyn_system}
\dot{\bm{x}} = A\bm{x} + B\bm{u}
\end{equation}
и  использует квадратичный функционал вида
	\begin{equation} \label{eq:lqr_cost_func}
	F = \int_0^{\infty}{(\bm{x}^T Q \bm{x} + \bm{u}^T R \bm{u})} dt
	\end{equation}
в качестве критерия оптимальности. Управление, минимизируещее функционал \eqref{eq:lqr_cost_func}, имеет вид
	\begin{equation} \label{eq:lqr_control_law}
	\bm{u} = -R^{-1} B^T P \bm{x},
	\end{equation}
	где $P$ находится из решения уравнения Риккати
	\begin{equation} \label{eq:riqatty}
	A^T P + P A - P B R^{-1} B^T P + Q = -\dot{P}.
	\end{equation}


Во второй части работы ~\cite{Bouabdallah01} авторы описывают синтез контура управления с использованием линейно-квадратичного регулятора. Для этого они линеаризуют модель около текущего положения объекта, затем решают уравнение Рикатти, используя метод Пирсона ~\cite{Longchamp01} и проводят численные и стендовые эксперименты. Результат оказался лучше, чем при применении ПИД-регуляторов, что частично объясняется использованием более полной модели.


Использование данного подхода для синтеза контура управления квадрокоптером достаточно подробно описано в статье исследователей из Мехико ~\cite{Reyes-Valeria01}. Модель аппарата была исполнена по \textit{plus}-схеме. Кинематика вращательного движения представлена в кватернионном описании. Вектор состояния содержит положение центра масс аппарата, его скорость, кватернион ориентации и угловую скорость. Динамика аппарата, помимо стандартных сил и моментов, содержит аэродинамические моменты, действующие на корпус БЛА. Для линеаризации модели авторы выбирают точку неподвижного зависания квадрокоптера и затем строят ЛКР-регулятор, который определяет оптимальную траекторию в его координатном пространстве. Авторы находят две различные матрицы усиления и разделяют управление на два независимых режима, действующих поочередно, когда аппарат находится вдали от целевой траектории и когда он переходит в режим слежения. Работоспособность контура управления иллюстрируется численными экспериментами. Графики демонстрируют сходимость траектории к целевой за некоторое время. В работе не приведены ошибки ориентации в процессе движения.

В работе ~\cite{Minh01} авторы расширяют модель системы \eqref{eq:linear_dyn_system}, добавив возмущения $G \bm{w}$ и шум измерений $\bm{v}$
\begin{equation} \label{eq:linear_dyn_system_noisy}
\begin{aligned}
&\dot{\bm{x}} = A\bm{x} + B\bm{u} + G \bm{w}\\
&\bm{y} = C \bm{x} + \bm{v}.
\end{aligned}
\end{equation}
Затем они используют расширенный фильтр Калмана для оценки состояния и строят линейно-квадратичное гауссовское (ЛКГ) управление с интегральным действием. Численные эксперименты демонстрируют возможность стабилизировать аппарат в точке статического зависания. Преймущество ЛКГ подхода заключается в отсутствии необходимости знать с абсолютной точностью текущее состояние контролируемого объекта для обеспечения оптимального управления.

\section{Управление с использованием скользящего режима}

Скользящий режим (СР) (\textit{sliding mode}) -- один из видов робастного управления, при котором управляющие воздейтвие на объект обеспечивает его движение в пределах выбранной поверхности в координатном пространстве, не позволяя выбранным параметрам выходить за пределы допустимых, чем обеспечивается устойчивость такого движения. При смещении траектории объекта за пределы поверхности включается активное управление, кторое возвращает его на одну из допустимых траекторий. В случае нелинейной динамики данный метод часто демонстрирует преймущества по сравнению с другими техниками из-за того, для его применения нет необходимости линеаризовывать уравнения движения.

В статье Xu, Ozguner ~\cite{Xu01} скользящий режим применяется для синтеза управления квадрокоптера. Авторы разделяют модель на две подсистемы -- полностью (независимо) управляемую и частично управляемую. Затем СР-подход применяется для второй системы, что позволяет разбить ее еще на несколько подсистем и синтезировать управление. В результате в численном эксперименте БЛА удается добраться до целевой точки и стабилизировать там свое положение за время около 20 секунд.

СР-подход также испоьзуется в работе ~\cite{Runcharoon01}. Чтобы уменьшить осциляции около поверхности, авторы вводят понятие граничного слоя включения. Аппарат способен стабилизировать свою позицию и угол рысканья с высокой точностью в условиях внешних возмущений.

Исследование ~\cite{Sumantri01}

\section{Управление с использованием метода обратной интеграции}

Метод обратной интеграции (\textit{backstapping}) -- рекурсивный алгоритм, основанный на разбиении динамической системы на набор подсистем и поочередной стабилизации каждой из них. Алгоритм не является вычислительно затратным и неплохо справляется с возмущениями, однако чувствителен к точности в оценках параметров модели. Его применение продемонстрировано в работе Madani и др. ~\ref{Madani01}. Авторы делят систему на три подсистемы, включающие полностью управляемую подсистему, частично управляемую подсистему и подсистему двигателей. Углы тангажа и крена удалось стабилизировать с использованием теории устойчивости Ляпунова. Аппарат за небольшое время приходит в заданное положение.

Похожий подход использовался разработчиками системы стабилизации ориентации БЛА в работе ~\cite{Huo01}. Используя анализ устойчивости Ляпунова авторы показали, что система асимптотически устойчива. Кватернионный подход к описанию ориентации помог избежать точек сингулярности и упростить поиск функции Ляпунова.

\section{Адаптивное управление}

Адаптивное управление -- класс алгоритмов, позволяющий синтезировать контур управления, позволяющие изменять параметры или структуру регулятора в зависимости от изменения параметров управляемого объекта или внешних возмущений. Примение подобных алгоритмов для квадрокоптеров чаще всего обусловлено наличием неопределенностей в измерении массово-габаритных и аэродинамических параметров аппарата.
Одним из примеров применения подхода является работа ~\cite{Huo01}. Авторы рассматривают квадрокоптер стандартной конструкции, выполненный по \textit{cross}-схеме. Ориентация представленна углами Эйлера. Синтезирован двухуровневый контур управления, первый уровень использует СР-контроллер для управления горизонтальной позицией БЛА. Второй уровень использует адаптивный регулятор для управления выделенной из модели полностью управляемой подсистемой, вектор состояния которой состоит из ориентации и общей тяги БЛА. Рассмотрено два подхода к построению адаптивного управления -- первый использует ошибку текущего положения и ориентации для оценки параметров динамики объекта, второй дополнительно использует разницу между измеренным и оцененным выходами управления. Комбинированный подход показал хорошую производительность в условиях полной неопределенности для ключевых параметров динамической модели.

Применение адаптивного управления для стабилизации движения квадрокоптера со смещающимся во времени центром масс рассмотрено в статье Palunko, Fierro ~\cite{Palunko01}. Авторы применили линеаризацию обратной связи с последующим применением ПД-регулятора и показали, что такое управление неспособно справиться со стабилизацией движния БЛА. Затем был применен аддитивный адаптивный компонент системы управления, с помощью которого оценивалось смещение центра масс аппарата и происходила коррекция рассчета управляющего вектора.

\section{Управление линеаризацией обратной связи}

Линеаризация обратной связи (ЛОС) подразумевает преобразование нелинейной системы в эквивалентную линейную с помощью замены переменных в модели. Помимо рассмотренной выше работы ~\cite{Palunko01}, такой подход использовали в ~\cite{Roza01} для синтеза контроллера, на вход которого подается траетория с возможностью построения скоростного профиля и управления углом рысканья в зависимости от пройденого пути. Численные эксперименты показали сходимость параметров движения БЛА к целевым.

Сравнение управления линеаризацией обратной связи с адаптивным СР управлением применительно к квадрокоптерам описано у ~\cite{Lee01}. ЛОС-конроллер показал неплохую производительность в отсутствии возмущений, однако оказался очень чуствителен к шуму измерений состояния и неучтенным в модели силам и моментам. Адаптивное управление показало себя лучше в условиях неопределенности.

Можно сделать вывод, что синтез контура управления с использованием линеаризации обратной связи можно эффективно применять совместно с другими алгоритмами, чтобы уменьшить влияние возмущений различной природы.

\section{Робастное управление}

Робастное управление подразумевает синтез регулятора, обеспечивающий качество управления при присутствии неточностей в модели управляемого объекта или параметров его динамики.

Bai ~\cite{Bai01} использует ПД-регулятор с робастным компенсатором для управления БЛА, исполненному по \textit{cross}-схеме. Такой синтез показывает хорошие результаты, несмотря на неопределенности в модели объекта.

В другой работе Tony и Mackunisy ~\cite{Tony01} синтезируют робастное управление, гарантирующее асимптотическую устойчивость в условиях ошибок в оценках параметров динамики системы. В результате получается эффективный и вычислительно простой алгоритм.

\section{Оптимальное управление}

Оптимальное управление основано на поиске регулятора, который обеспечит минимизацию целевого функционала, сформулированного с учетом контекста задачи. Для БЛА целью оптимального управления может быть минимизация расхода энергии, максимизация длительности или дальности полета, построение оптимальных траекторий и другие. Обычно оптимальный контроллер не отличается робастностью и для его успешного синтеза требуется хорошо знать параметры системы. 


Синтез $L_1$ оптимального регулятора, обладающего относительно неплохими робастными качествами описан в работе ~\cite{Satici01}. Авторам удалось минимизировать негативные эффекты, возникающие в следствие шумов измерения текущего состояния и возмущений без их непосредственного измерения.

$H_{\infty}$ оптимальное управление было применено к упрощенной модели динамики квадрокоптера в статье Falkenberg и др. ~\cite{Falkenberg01}. Алгоритм показал очень высокую производительность даже в условиях сильных возмущений.

Для управления ориентацией БЛА и отслеживания целевых траекторий в работе Raffo и др. ~\cite{Raffo01} был синтезирован интегральный $H_{\infty}$ оптимальный регулятор. Численные эксперименты показали хорошую сопротивляемость помехам и возмущением. Интегральная составляющая имела важное значение для качества управления.

\section{Управление с использованием искуственных нейронных сетей и алгоритмов нечеткой логики}

Применение искуственных нейронных сетей и алгоритмов нечеткой логики (\textit{fuzzy logic}) в приложении к системам управления мультироторными роботами в последнее время стало популярным и продемонстрировало свою эффективность. Нечеткая логика  -- обобщение класичиской теории множеств и логики, в основе которого лежит понятие нечеткого множества, где функция принадлежности элемента к множеству не является бинарной. Примером применения алгоритмов нечеткой логики является работы ~\cite{Dierks01} и ~\cite{Santos01}. Авторы интерпритируют текущую ошибку положения и ориентации БЛА как элемент нечеткого множества, затем синтезируется ПИД-регулятор. Численные эксперементы демонстрируют принципиальную возможность управления квадрокоптером, применяя данные алгоритмы.


Искуственные нейроныые сети -- алгоритм, основанный на подборе параметров (обучении) функции специального вида (нейросети) таким образом, чтобы эта функция вела себя желательным образом на рассматриваемой множестве. Таким образом, алгоритм можно применить для управления системами; в качестве входа может быть рассмотрена ошибка состояния, а в качестве выхода -- сигнал управления, который будет устранять эту ошибку. Подобным образом поступили \textit{Nicol, Macnab, Ramirez-Serrano} ~\cite{Nicol01}. Авторам удалось спроектировать систему таким образом, чтобы она была асимптотически устойчивой.


Методы глубинного обучения были применены Andersson и др. ~\cite{Andersson01}. Авторы обучили нейронную сеть с помощью самостоятельно созданного алгоритма, оптимизирующего траектории в координатном пространстве, учитывающий неточности модели, ограничения на максимальную силу тяги роторов и динамические препятствия. Затем они показали пример успешного применения алгоритма при значительно меньших затрачиваемых вычислительных ресурсах относительно современного бортового контроллера.

\section{Квадрокоптеры с расширенным вектором управляющего воздействия}

Для увеличения маневренных характеристик квадрокоптеров и повышения их общей управляемости многие исследователи придпринимали попытки усовершенствовать конструкцию БЛА. Одним из вариантов таких изменений является добавление специальных приводов, которое бы меняли геометрию пропеллеров, тем самым значительно увеличивая пределы и скорость измнения их тяги или оптимизируя потребляемую в полете энергию.


В работе \textit{Cutler} и др. ~\cite{Cutler01} сравниваются возможности квадракоптеров с фиксированной и изменяющейся геометрией пропеллеров. Сначала описывается расширенная модель двигателей и пропеллеров, которая затем линеаризуется. Затем авторы описывают исследовательский стенд и реализуют несколько вариантов управления с переменной геометрией пропеллера, которые наглядно демонстрируют возможность экономить в полете потребляемую энергию и выполнять сложные маневры, недоступные стандартному квадрокоптеру, например, зависать в перевернутом положении. Продолжение данного исследования приведено в ~\cite{Cutler02}, где авторы принимают во внимание ограничения на выходы приводов БЛА и строят траектории с учетом этих ограничений.

Другой распространенный способ расширить размерность вектора управляющего воздействия БЛА -- добавить возможность изменять направления тяги его двигателей. Такой квадрокоптер будет называться тильтротор (tilt rotor), его динамике и управлению посвящено много работ, среди которых большая часть было опубликовано в последние годы.

Одну из реализаций контура управления для тильтротора предложили \textit{Ryll} и др. ~\cite{Ryll01}. Исследователи рассматривают квадрокоптер \textit{cross}-схемы c двигателями, которые могут вращаться вокруг лучей прикрепления. Модель динамики аппарата рассматривает основные внешние силы и моменты, действующие на аппарат а также подробно описывают взаимодействие каждого из поворотных роторов с корпусом. Цель управления -- обеспечение движения центра масс аппарата по целевой траектории при независимом управлении ориентацией корпуса. В вектор состояния входят координаты ценра масс аппарата и его ориентация, представленная в матричном виде. В качестве компонент вектора управляющего воздействия выбраны скорости вращения пропеллеров вокруг вертикальных осей двигателей и скорости поворота двигателей вокруг соответсвующих лучей. Для синтеза контура управления авторы упрощают динамическую модель, принебрегая некоторыми эффектами, затем дифференцируют уравнения движения по времени и получают линейную относительно вектора управления систему. Затем они обращают динамику, используя псевдообращение Мура-Пенроуза. При этом избыточные степени свободы системы используются для того, чтобы гарантировать условия, при которых это обращение возможно. Для того, чтобы обеспечить экспоненциальную сходимость траектории к целевой, авторы синтезируют регулятор в который входят пропорциональная, дифференциальная и дважды дифференциальная части. Сложности с измерениями второй производной положения и ориентации авторы предлагают решать с помощью их оценки по выходу из модели, измеряя скорости вращения двигателей и их наклоны. Для уточнения модели авторы проводят ряд измерений, экспериментально уточняя параметры динамики пропеллеров и переходные процессы в сервоприводах, отвечающих за наклон двигателей. После численных экспериментов исследователи демонстрируют воплощение алгоритмов в прототипе. Для оценки его состояния используются внешниие камеры. В качестве траектории выбрана плоская "восьмерка". Ошибка позиционирования составила около 5 сантиметров, ориентации -- не более 0,1 радиана.

Некоторый анализ и попытки расширения работы \textit{Ryll} предприняли Шольц, Троммер ~\cite{Stolc01}.

В работе  ~\cite{Invernizzi01} рассматривается управляемая динамика квадрокоптера с поворотными роторами. Использован весьма оригинальный подход  и синтезу контура управления с использований ряда геометрических приобразований, для чего пришлось значительно упростить математическую модель БЛА. Для учета ограничений на максимальные углы отклонения поворотных роторов авторы запретили выходить вектору общей тяги аппарата из конусовидной области, параметры которой опредяляются пределами отклонения сервоприводов. Численные эксперименты показали способность аппарата отслеживать траекторию в виде плоской восьмерки, одновременно меняя свою ориентацию. При этом углы отклонения сервоприводов не вышли за пределы обозначенных лимитов, однако, как сами пояснили авторы, они еще не смогли доказать, что применение такого метода всегда будет гарантировать подобный результат.

Еще один пример использования поворотных роторов в конструкции квадрокоптера -- работа \textit{Nemati} и др. ~\cite{Nemati01}. Авторы рассматривают квадрокоптер \textit{plus}-конфигурации. Модель подразумевает возможность симметрично поворачиваться одной паре двигателей. Авторы применяют метод линеаризаци обратной связи и показывают, что таким образом возможно осуществлять движение и независимо управлять углом тангажа.


Другую техникку для синтеза управления квадрокоптера с поворотными роторами применили \textit{Falconi}, \textit{Melchiorri} ~\cite{Falconi01}. Ловкая замена переменных в модели и применение псевдообращения Мура-Пенроуза помогли им обратить динамику БЛА и получить выражения для углов отклонения двигателей и оборотов каждого из них, чтобы выход системы соостветсвовад выходу ПД регулятора, который обеспечивает сходимость положения и ориентации БЛА к целевым. Вычислительные эксперименты показали работоспособность предложенного алгоритма.

Скользящий режим для синтеза контура управления БЛА с поворотными роторами применил Yih ~\cite{Yih01}. Контур содержал два скользящих контроллера для повышения надежности управления. Было показано, что алгоритм является ассимптотичекки устойчивым. Численные эксперименты подтвердили способность аппарата перемещаться в точку не изменяя ориентацию корпуса.

Исследование маневренных возможностей тильтроторов с большими лимитами выполняющих поворототы двигателей сервоприводов исследовал \textit{Oosedo} и др. ~\cite{Oosedo01}. Показана возможность полета с углами тангажа и рысканья окло 90 градусов. Проведены летные испытания.

Вычислительно простой способ управления БЛА с поворотными роторами разработан \textit{Alkamachi} ~\cite{Alkamachi01}. Рассмотрена нестандартная схема, где каждый из двигателей может вращаться вокруг одной оси, параллельной поперечной оси корпуса. Для управления позицией и ориентацией использовались ПИД-регуляторы. Все роторы наклонялись синхронно. Контур управления показал хорошую производительность в условиях внешних возмущений, шума измерений и неопределенности в параметрах динамики.
